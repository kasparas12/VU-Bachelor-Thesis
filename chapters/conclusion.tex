\sectionnonum{Išvados}

Išanalizavus rezultatus, pateikiamos šios \textbf{išvados}:

\begin{enumerate}
    \item Atsižvelgiant į tyrimo metu nustatytą peržiūros roboto darbo pagreitėjimą, galima daryti išvadą, kad esamas debesų kompiuterijos tiekėjų siūlomų paslaugų lygis ir kainodara užtikrina galimybę efektyviai vykdyti nedidelio-vidutinio lygio automatinį saityno peržiūros procesą, todėl:
    \begin{itemize}
        \item tokios realizacijos robotai turėtų būti kategorizuoti atlikti temines peržiūras, akcentuojančias turinio kokybę, o ne aprėptį;
        \item plačiųjų, inkrementinių saityno peržiūrų vykdymas šiomis aplinkybėmis nėra įmanomas.
    \end{itemize}
    
    \item Javascript variklio naudojimas saityno peržiūros roboto darbo ciklo metu yra brangi operacija, todėl kiekvieno resurso žvalgymas pasitelkiant tokias naršyklių tvarkykles yra neefektyvus 
    
    \item Šiuo metu nėra universalios metodikos atpažinti faktą, jog svetainė naudoja dinaminį HTML turinio atvaizdavimą
    
    \item Remiantis tyrimo metu nustatytu vidutiniu atrandamų naujų URL adresų skaičiu taikant populiariųjų karkasų ir bibliotekų atpažinimo metodiką, galima daryti išvadą, kad metodas užtikrina gana efektyvų \textbf{tikimybinį} modelį atlikti žiniatinklio peržiūros procesą įtraukiant ir modernias, dinaminį atvaizdavimą naudojančias svetaines, tačiau:
    \begin{itemize}
        \item metodas išlieka tikimybiniu -- egzistuoja reali galimybė neaplankyti resurso, kuris naudoja dinaminį atvaizdavimą, tačiau tam pasitelkia specializuotas, nepopuliarias bibliotekas ar karkasus;
        \item taikant modelį privaloma atlikti svetainės naudojamų skriptų turinio analizę -- tam sunaudojamas didelis tinklo srautas, todėl efektyvi kešavimo ir kontrolinės sumos skaičiavimo strategija yra būtina;
    \end{itemize}
\end{enumerate}

\textbf{Galimos ateities darbų kryptys:}

