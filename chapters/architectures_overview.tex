\section{Peržiūros roboto architektūra}

Šiame skyriuje analizuojamos mokslinėje literatūroje aprašytos saityno peržiūros robotų realizacijos -- pagrindiniai komponentai, jų funkcinės atsakomybės, charakteristikos, atliekama palyginamoji analizė tarp nagrinėjamų peržiūros robotų.

\subsection{Komponentai}

M.Najorc ir C. Olston mokslinė saityno peržiūros sistemų robotų apžvalga (\cite{StanfWebCrawl}) formalizuoja anksčiau literatūroje aprašytas tokių sistemų dizaino specifikas. Joje nusakoma išskirstyto peržiūros roboto architektūra -- skirtingose mašinose egzistuojantys žvalgymo procesai, kiekvienas jų turintis keletą lygiagrečiai veikiančių agentų gijų, kurios atlieka kartotinius žvalgymo ciklo žingsius, kuriuose dalyvauja išskiriami pagrindiniai 8 struktūriniai sistemos komponentai.

\subsubsection{„Pasienis“}

„Pasienio“ duomenų struktūra (angl. -- \textit{URL Frontier}) saugo URL\footnote{URL - Uniform Resource Locator} adresų sąrašą, kurie bus aplankyti, iš šio sąrašo paduodamas adresas žvalgymo agento gijai pagal atitinkamas žvalgymo mandagumo (angl. -- \textit{Politeness}) ir prioritizavimo (angl. -- \textit{Priority}) politikas (\cite{StanfWebCrawl}. Tai viena iš pagrindinių žvalgymo roboto būsenos duomenų struktūrų. Jai keliami šie pagrindiniai funkcionalumo reikalavimai:
\begin{itemize}
    \item Pridėti URL adresą į sąrašą
    \item Nuskaityti URL adresą iš sąrašo
\end{itemize}

\subsubsection{HTTP parsiuntimo modulis}

Žvalgymo agentui gavus URL adresą iškviečiamas HTTP modulis, kuris pirmiausia kreipiasi į \textit{DNS adreso išaiškinimo} komponentą tam, jog būtų nustatytas URL resurso serverio vardo IP protokolo adresas \cite{StanfWebCrawl}. Šis veiksmas reikalingas tam, kad būtų minimizuotas HTTP užklausos atsakymo laikas (išvengiama DNS išaiškinimo užklausų į išorinius serverius).

\subsubsection{Saityno nuorodų ištraukiklis}

Šis komponentas (angl. -- \textit{Link Extractor}) nuskaito parsiųsto HTML dokumento turinį ir išgauna visas HTML nuorodas tiek į išorinius (angl. -- \textit{Offsite Links}), tiek į vidinius (\textit{In-site Links}) žiniatinklio serverio puslapius \cite{StanfWebCrawl}.

\subsubsection{Adresų skirstiklis}

Šis modulis (angl. -- \textit{URL Distributor}) atsakingas už išgautų nuorodų priskyrimą atitinkamiems žvalgymo procesams \cite{StanfWebCrawl}.

\subsubsection{Adresų filtras}

Komponentas, kuris filtruoja priskirtus URL adresus ir gali išmesti taisyklių neatitinkančias nuorodas (pvz.: puslapiai, įtraukti į juodąjį sąrašą) \cite{StanfWebCrawl}. Taisyklės gali būti specializuotos kiekvienam žvalgymui atskirai.

\subsubsection{Dublikatų šalintojas}

Roboto dalis, kuri atlieką testą, ar URL nuoroda dar nebuvo aplankyta peržiūros metu, pagrindiniai keliami funkcionamulo reikalavimai \cite{StanfWebCrawl}:
\begin{itemize}
    \item Pridėti URL adreso aplankymo indikatorių į sąrašą
    \item Atlikti URL priklausymo sąrašui testą
\end{itemize}

\subsubsection{Adresų prioritizuotojas}

Komponentas (angl. -- \textit{URL Prioritizer}), kuris kiekvienam URL adresui priskiria tam tikrą prioritetą pagal specializuotus saityno peržiūros roboto sistemos pasirinkimo politikos faktorius, tokius kaip nustatomas puslapio svarbos laipsnis ar puslapio keitimosi greičio faktorius \cite{StanfWebCrawl}.


\subsection{Peržiūros vykdymo ciklas}

Atsivelgus į 3.1 poskyrio struktūrinius komponentus pagal \cite{StanfWebCrawl} pasiūlytą schematinį dizainą, galima sudaryti veiklos diagramą, parodančią sistemos ciklinį funkcionavimą.

\begin{figure}[ht!]
\centering
\includegraphics[scale=0.5]{img/Web_Crawler_Activity_Diagram.png}
\caption{Saityno žvalgymo roboto sistemos UML veiklos diagrama \cite{CategoriesOfWebCrawlersAndOverview}}
\label{fig:system_activity_diagram}
\end{figure}

\subsection{Literatūroje aprašytų robotų palyginamoji analizė}

Šiame skyriuje lyginama keletas žinomiausių akademiniuose šaltiniuose aprašytų saityno peržiūros robotų architektūrų siekiant suvokti, kaip kiekviena jų koordinuoja peržiūros proceso agentų darbą.

\subsubsection{Peržiūros robotų išplečiamumo iššūkis}

Nors koncepcinis peržiūros sistemų algoritmas, aprašytas 1 skyriuje, yra labai paprastas, tokių sistemų problemos kyla sprendžiant išplėtimo iššūkius -- siekiant peržiūrėti milijardus svetainių per pagrįstai trumpą laiką ir išlaikyti peržiūrėtų svetainių naujausią galimą kopiją \cite{WCArchitectureMicrosoft}. Pagal \cite{WCArchitectureMicrosoft} apžvalgą, galima būtų iškelti šį pagrindinį peržiūros robotų architektūrų iššūkį:

\begin{itemize}
  \item Sugebėti vykdyti peržiūros procesą išskirstytai ir lygiagrečiai, tačiau vykdyti tai etiškai neapkraunant žiniatinklio serverio (etiško žvalgymo politika)
\end{itemize}

\subsubsection{„PolyBot“ sistema}

V. Shkapenyuk ir T. Suel aprašytas tinkle skirtinguose mazguose išskirstytas saityno peržiūros robotas, parašytas C++ ir Python programavimo kalbomis \cite{PolyBotArchitecture}.

\subsubsubsection{Peržiūros išskirstymas ir lygiagretumas}

Aprašyta architektūra pasižymi peržiūros valdiklio komponentu, kuris gauna URL adresų užklausas ir paskirsto jas parsiuntimo komponentams (Python komponentams) \cite{PolyBotArchitecture}. Kiekvienas valdiklis gali komunikuoti daugiausiai su 8 parsiuntimo moduliais \cite{PolyBotArchitecture}. Peržiūros valdikliai tokioje architektūroje sukelia „butelio kaklelio“ efektą.

\subsubsubsection{Puslapio turinio dublikatų testas}

Aprašytoje architektūroje nerealizuotas

\subsubsubsection{Priklausymo URL sąrašui testas}

Operatyviojoje atmintyje Red-Black medžio struktūros pavidalu saugomas aplanytų adresų sąrašas, kuris, pasiekus limitą, perkeliamas į disko atmintį \cite{PolyBotArchitecture}. Siekiant išvengti dažnų disko rašymo/skaitymo operacijų, šios procedūros atliekamos paketais.

\subsubsubsection{„Pasienio“ komponento strukūra}
\\ TODO.


\subsubsection{„Mercator“ sistema}

1999 m. išsamiai literatūroje autorių A.Heydon ir M.Najork aprašyta ir eksperimentiškai įvertinta paskirstyta saityno peržvalgos roboto architektūra. Ši sistema parašyta naudojantis Java programavimo kalba ir jos vykdomąja aplinka (angl. -- \textit{runtime environment}). Aprašytas robotas didžiąją dalį savo aprašomų duomenų struktūrų talpina disko atmintyje, taip pat nedideli buferiai saugomi operatyvioje atmintyje ir skirti pasiekti greitesnei duomenų prieigai (kešavimo mechanizmas) \cite{MercatorLiterature}.

\subsubsubsection{Peržiūros išskirstymas ir lygiagretumas}

Sistemoje saityno peržiūra vyksta naudojant žvalgymo procesų gijas (angl. -- \textit{Worker Threads}, kurios veikia lygiagrečiai ir įprastai sistemoje skaičiuojamos šimtais vienetų vienu metu \cite{MercatorLiterature}. Kiekviena gija atsakinga už puslapio atsiuntimą ir apdorojimą (nuorodų suradimą, suabsoliutinimą) Nors fizinės mašinos lygmenyje veikia daug gijų, tačiau sprendimas neparemtas topologiniu peržiūros išskaidymu \cite{MercatorLiterature}. Šioje architektūroje aprašomas peržiūros robotas yra centralizuotas -- egzistuoja peržiūros koordinatoriaus komponentas, kuris atsitiktiniu I/O būdu komunikuoja su lygiagrečiose gijose veikiančiais peržiūros agentais ir renka jų informaciją \cite{MercedCloudBasedWebCrawler}.

\subsubsection{„Heritrix“ sistema}
Nepelno siekiančios organizacijos „Internet Archive“ suprojektuotas peržiūros robotas, kurio pagrindinis tikslas rinkti svetainių kopijas ir saugoti jų registrą ateities kartoms. \cite{HeritrixArchitecture}.

\ref{fig:heritrix} schemoje pavaizduota šio roboto architektūra, pasižyminti 3 esminiais komponentais -- operatoriaus konsole (asministratoriaus valdymo sąsaja), peržiūros konfigūratoriumi (peržiūros parametrų nustatymas) ir peržiūros valdikliu (peržiūros proceso koordinavimas) \cite{HeritrixArchitecture}. Sistema nepalaiko išskirstyto tinkle peržiūros koordinavimo \cite{HeritrixArchitecture}, žvalgymas vyksta vienoje fizinėje mašinoje, programa turi daug peržiūros agentų gijų, dirbančių lygiagrečiai \cite{HeritrixArchitecture}. Šioms gijoms duodami žvalgytini URL adresai kitas(URL) operacijos pagalba. Šį koordinavimo procesą atlieka peržiūros valdiklis \cite{HeritrixArchitecture}.
\begin{figure}[htp!]
\hspace{-1cm}
\centering
\includegraphics[scale=0.6]{img/heritrix.png}
\caption{„Heritrix“ peržiūros roboto architektūra}
\label{fig:heritrix}
\end{figure}

\pagebreak

Iš \ref{tab:architectures_comparisonl} lentelės apžvelgtų literatūroje pateikiamų klasikinių peržiūros robotų palyginimo matoma, jog aprašyti sprendimai plačiai taiko centralizuotą peržiūros modelį (koordinatorius), kuris didinant lygiagrečiai veikiančių agentų skaičių lemia ribotą greitaveiką, nes yra bendrinis visiems agentams. Taip pat pastebima, jog didžioji dauguma sprendimų nepalaiko išskirstytos tinkle peržiūros koordinacijos (nebent atskiri peržiūros procesai), išskyrus „PolyBot“, kuris teoriškai tą palaiko, tačiau pagal \cite{PolyBotArchitecture} pateikiamą literatūrinę medžiagą, tokia konfigūracija niekada nebuvo išbandyta praktiškai.
\subsubsection{Palyginimo lentelė}

\begin{table}[htbp]
  \centering
  \caption{Nagrinėtų architektūrų palyginimas}
    \begin{tabular}{|l|l|l|l|l|}
    \toprule
    \textbf{Robotas} & \textbf{Kalba} & \multicolumn{1}{p{7.61em}|}{\textbf{Išskirstytos tinkle peržiūros koordinavimas}} & \multicolumn{1}{p{8em}|}{\textbf{Išskirstyto žvalgymo modelis}} & \multicolumn{1}{p{9.555em}|}{\textbf{Koordinatoriaus egzistavimas}} \\
    \midrule
    „PolyBot“ & C++, Python & Palaikoma & Lygiagrečios gijos & Taip \\
    \midrule
    „Mercator“ & Java  & Nepalaikoma & Lygiagrečios gijos & Taip \\
    \midrule
    „Heritrix“ & Java  & Nepalaikoma & Lygiagrečios gijos & Taip \\
    \bottomrule
    \end{tabular}%
  \label{tab:architectures_comparisonl}%
\end{table}%
