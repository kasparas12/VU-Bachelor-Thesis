\section{Peržiūros robotų architektūros}

Šiame skyriuje analizuojamos mokslinėje literatūroje aprašytos saityno peržiūros roboto sistemų realizacijos -- pagrindiniai komponentai, jų funkcinės atsakomybės, charakteristikos, atliekama palyginamoji analizė tarp skirtingų aprašytų realizacijų.

\subsection{Sistemos komponentai}

M.Najorc ir C. Olston mokslinė saityno peržiūros sistemų apžvalga (\cite{StanfWebCrawl}) formalizuoja anksčiau literatūroje aprašytas tokių sistemų dizaino specifikas. Joje nusakoma išskirstyto peržiūros roboto sistemos architektūra -- skirtingose mašinose egzistuojantys žvalgymo procesai, kiekvienas jų turintis keletą lygiagrečiai veikiančių agentų gijų, kurios atlieka kartotinius žvalgymo ciklo žingsius, kuriuose dalyvauja išskiriami pagrindiniai 8 struktūriniai sistemos komponentai.

\subsubsection{„Pasienis“}

„Pasienio“ duomenų struktūra (angl. -- \textit{URL Frontier}) saugo URL\footnote{URL - Uniform Resource Locator} adresų sąrašą, kurie bus aplankyti, iš šio sąrašo paduodamas adresas žvalgymo agento gijai pagal atitinkamas žvalgymo mandagumo (angl. -- \textit{Politeness}) ir prioritizavimo (angl. -- \textit{Priority}) politikas. Tai viena iš pagrindinių žvalgymo roboto sistemos būsenos duomenų struktūrų. Jai keliami šie pagrindiniai funkcionalumo reikalavimai:
\begin{itemize}
    \item Pridėti URL adresą į sąrašą
    \item Nuskaityti URL adresą iš sąrašo
\end{itemize}

\subsubsection{HTTP parsiuntimo modulis}

Žvalgymo agentui gavus URL adresą iškviečiamas HTTP modulis, kuris pirmiausia kreipiasi į \textit{DNS adreso išaiškinimo} komponentą tam, jog būtų nustatytas URL resurso serverio vardo IP protokolo adresas \cite{StanfWebCrawl}. Šis veiksmas reikalingas tam, kad būtų minimizuotas HTTP užklausos atsakymo laikas (išvengiama DNS išaiškinimo užklausų į išorinius serverius).

\subsubsection{Saityno nuorodų ištraukiklis}

Šis komponentas (angl. -- \textit{Link extractor}) nuskaito parsiųsto HTML dokumento turinį ir išgauna visas HTML nuorodas tiek į išorinius (angl. -- \textit{offsite links}), tiek į vidinius (\textit{in-site links}) žiniatinklio serverio puslapius.

\subsubsection{Adresų skirstiklis}

Šis modulis (angl. -- \textit{URL distributor}) atsakingas už išgautų nuorodų priskyrimą atitinkamiems žvalgymo procesams.

\subsubsection{Adresų filtras}

Komponentas, kuris filtruoja priskirtus URL adresus ir gali išmesti taisyklių neatitinkančias nuorodas (pvz.: puslapiai įtraukti į juodąjį sąrašą). Taisyklės gali būti specializuotos kiekvienam žvalgymui atskirai.

\subsubsection{Dublikatų šalintojas}

\begin{itemize}
    \item Pridėti URL adreso aplankymo indikatorių į sąrašą
    \item Atlikti URL priklausymo sąrašui testą
\end{itemize}

\subsubsection{Adresų prioritizuotojas}

Komponentas (angl. -- \textit{URL Prioritizer}), kuris kiekvienam URL adresui priskiria tam tikrą prioritetą pagal specializuotus saityno peržiūros roboto sistemos pasirinkimo politikos faktorius, tokius kaip nustatomas puslapio svarbos laipsnis ar puslapio keitimosi greičio faktorius.


\subsection{Peržiūros vykdymo ciklas}

Atsivelgus į 3.1 poskyrio struktūrinius komponentus pagal \cite{StanfWebCrawl} pasiūlytą schematinį dizainą, galima sudaryti veiklos diagramą, parodančią sistemos ciklinį funkcionavimą.

\begin{figure}[ht!]
\centering
\includegraphics[scale=0.5]{img/Web_Crawler_Activity_Diagram.png}
\caption{Saityno žvalgymo roboto sistemos UML veiklos diagrama \cite{CategoriesOfWebCrawlersAndOverview}}
\label{fig:system_activity_diagram}
\end{figure}

\subsection{Literatūroje aprašytų robotų palyginamoji analizė}

Šiame skyriuje apžvelgiamos keletas žinomiausių viešuose akademiniuose literatūros šaltiniuose aprašytų saityno peržiūros robotų sistemų architektūrų ir siekiama palyginti dizaino sprendimus įvertinant stipriąsias, silpnąsias tokių sistemų puses, rasti bendras šio tipo sistemų komponentes.

\subsubsection{„Mercator“ sistema}

1999 m. išsamiai literatūroje autorių A.Heydon ir M.Najork aprašyta ir eksperimentiškai įvertinta paskirstyta saityno peržvalgos roboto architektūra. Ši sistema parašyta naudojantis Java programavimo kalba ir jos vykdomąja aplinka (angl. -- \textit{runtime environment}). Aprašytas robotas didžiąją dalį savo aprašomų duomenų struktūrų talpina disko atmintyje, taip pat nedideli buferiai saugomi operatyvioje atmintyje ir skirti pasiekti greitesnei duomenų prieigai (kešavimo mechanizmas) \cite{MercatorLiterature}.

\subsubsubsection{Peržiūros išskirstymas ir lygiagretumas}

Sistemoje saityno peržiūra vyksta naudojant žvalgymo procesų gijas (angl. -- \textit{Worker Threads}, kurios veikia lygiagrečiai ir įprastai sistemoje skaičiuojamos šimtais vienetų vienu metu \cite{MercatorLiterature}. Kiekviena gija atsakinga už puslapio atsiuntimą ir apdorojimą (nuorodų suradimą, suabsoliutinimą) Nors fizinės mašinos lygmenyje veikia daug gijų, tačiau sprendimas neparemtas topologiniu peržiūros išskaidymu \cite{MercatorLiterature}. Šioje architektūroje aprašomas peržiūros robotas yra centralizuotas -- egzistuoja peržiūros koordinatoriaus komponentas, kuris atsitiktiniu I/O būdu komunikuoja su lygiagrečiose gijose veikiančiais peržiūros agentais ir renka jų informaciją \cite{MercedCloudBasedWebCrawler}.


\subsubsection{„PolyBot“ sistemos architektūra}
\subsubsection{„BUbiNG“ didelio masto žvalgymo sistemos architektūra}