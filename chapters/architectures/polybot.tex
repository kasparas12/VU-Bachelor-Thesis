\subsubsection{„PolyBot“ sistema}

V. Shkapenyuk ir T. Suel aprašytas tinkle skirtinguose mazguose išskirstytas saityno peržiūros robotas, parašytas C++ ir Python programavimo kalbomis \cite{PolyBotArchitecture}.

\subsubsubsection{Peržiūros išskirstymas ir lygiagretumas}

Aprašyta architektūra pasižymi peržiūros valdiklio komponentu, kuris gauna URL adresų užklausas ir paskirsto jas parsiuntimo komponentams (Python komponentams) \cite{PolyBotArchitecture}. Kiekvienas valdiklis gali komunikuoti daugiausiai su 8 parsiuntimo moduliais \cite{PolyBotArchitecture}. Peržiūros valdikliai tokioje architektūroje sukelia „butelio kaklelio“ efektą.

\subsubsubsection{Puslapio turinio dublikatų testas}

Aprašytoje architektūroje nerealizuotas

\subsubsubsection{Priklausymo URL sąrašui testas}

Operatyviojoje atmintyje Red-Black medžio struktūros pavidalu saugomas aplanytų adresų sąrašas, kuris, pasiekus limitą, perkeliamas į disko atmintį \cite{PolyBotArchitecture}. Siekiant išvengti dažnų disko rašymo/skaitymo operacijų, šios procedūros atliekamos paketais.

\subsubsubsection{„Pasienio“ komponento strukūra}
\\ TODO.

