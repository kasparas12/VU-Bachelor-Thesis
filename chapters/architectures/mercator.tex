\subsubsection{„Mercator“ sistema}

1999 m. išsamiai literatūroje autorių A.Heydon ir M.Najork aprašyta ir eksperimentiškai įvertinta paskirstyta saityno peržvalgos roboto architektūra. Ši sistema parašyta naudojantis Java programavimo kalba ir jos vykdomąja aplinka (angl. -- \textit{runtime environment}). Aprašytas robotas didžiąją dalį savo aprašomų duomenų struktūrų talpina disko atmintyje, taip pat nedideli buferiai saugomi operatyvioje atmintyje ir skirti pasiekti greitesnei duomenų prieigai (kešavimo mechanizmas) \cite{MercatorLiterature}.

\subsubsubsection{Peržiūros išskirstymas ir lygiagretumas}

Sistemoje saityno peržiūra vyksta naudojant žvalgymo procesų gijas (angl. -- \textit{Worker Threads}, kurios veikia lygiagrečiai ir įprastai sistemoje skaičiuojamos šimtais vienetų vienu metu \cite{MercatorLiterature}. Kiekviena gija atsakinga už puslapio atsiuntimą ir apdorojimą (nuorodų suradimą, suabsoliutinimą) Nors fizinės mašinos lygmenyje veikia daug gijų, tačiau sprendimas neparemtas topologiniu peržiūros išskaidymu \cite{MercatorLiterature}. Šioje architektūroje aprašomas peržiūros robotas yra centralizuotas -- egzistuoja peržiūros koordinatoriaus komponentas, kuris atsitiktiniu I/O būdu komunikuoja su lygiagrečiose gijose veikiančiais peržiūros agentais ir renka jų informaciją \cite{MercedCloudBasedWebCrawler}.
