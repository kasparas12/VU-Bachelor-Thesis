% Lithuanian version


\sectionnonumnocontent{Santrauka}

Šiame darbe nagrinėjami saityno peržiūros robotai -- programų sistemos, dažniausiai naudojamos paieškos sistemose naujam saityno turiniui surasti, skirtos atlikti automatinę rekursyvią žiniatinklio resursų peržiūrą keliaujant URL adresų erdvės grafu. Siekiama įvertinti, ar tokių sistemų įgyvendinimas remiantis esamomis debesų kompiuterijos infrastruktūros paslaugomis yra efektyvus ir optimalus kaštų prasme. Taip pat sprendžiama dinamiškai HTML turinį generuojančių svetainių peržiūros problema. 

Darbe nagrinėjama peržiūros robotų probleminė sritis: apibrėžiamos tokių sistemų veikimo ribos, atliekama literatūroje egzistuojančių peržiūros robotų architektūrinių sprendimų analizė, identifikuojami skirtingi žiniatinklio svetainių atvaizdavimo modeliai ir jų įtaka roboto peržiūros procesui. Praktinėje darbo dalyje remiantis išanalizuota teorine medžiaga realizuojamas debesų kompiuterijos sprendimais paremtas peržiūros roboto prototipas ir įvertinamos jo peržiūros efektyvumo galimybės.

Nustatyta, kad taikyti išplėstines, į turinio aprėptį orientuotas peržiūras, naudojantis debesų kompiuterijos paslaugomis paremta architektūra nėra įmanoma, tačiau mažo-vidutinio lygio peržiūroms, kurios orientuotos į atrandamo turinio kokybę, tokios architektūros gali būi efektyvios. Nors darbo metu įsitikinta, kad neegizstuoja universalus būdas identifikuoti dinamiškai HTML turinį generuojančias svetaines, pristatytas efektyvus tikimybinis modelis, leidžiantis išgauti reikšmingą skaičių naujų, neišžvalgytų URL adresų.  
\raktiniaizodziai{saityno peržiūra, saityno inforamcijos rinkimas, saityno peržiūros robotas, išskirstyta sistema, debesų kompiuterija, klientinės pusės atvaizdavimas, dinaminis HTML turinio atvaizdavimas}   

% English version

\sectionnonumnocontent{Summary}

//TODO: summary in EN

\keywords{Web Crawling, Web Scraping, Web Crawling Robot, Distributed System, Cloud Computing, Client-Side Rendering, JavaScript Rendering}