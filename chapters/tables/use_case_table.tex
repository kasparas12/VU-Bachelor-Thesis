% Table generated by Excel2LaTeX from sheet 'crawling_vs_scraping'
\begin{table}[ht]
  \centering
  \caption{Panaudojamumo atvejų paaiškinimai}
    \begin{tabular}{|p{2em}|p{13em}|p{20em}|}
    \hline
    \textbf{Id} & \textbf{Pavadinimas} & \textbf{Aprašymas} \bigstrut\\
    \hline
    U1.1 & Inicijuoti peržiūros procesą & Perkelti pradinį peržiūros URL adresų sąrašą į peržiūros eilę prieš tai patikrinti, ar URL adresai dar neperžiūrėti, ir sukurti pirmąjį peržiūros agentą, jam priskirti domeno vardo zoną \\
    \hline
    U1.2 & Stabdyti peržiūros procesą & Panaikinti visus aktyvius peržiūros robotus ir jų peržiūros eiles  \\
    \hline
    U1.3 & Keisti aktyvių peržiūros agentų skaičių & Padidinti arba pamažinti maksimalų leistiną peržiūros ir Javascript atvaizdavimo agentų kiekį sistemoje \\
    \hline
    U2 & Rekursyviai vykdyti puslapio peržiūrą & Aplankyti URL adresu nurodytą svetainę, išnagrinėti jos HTML turinį ir išgauti visas nuorodas, patikrinti, ar jos dar neperžiūrėtos, perduoti jas atitinkam peržiūros robotui  \\
    \hline
    U2.1 & Įvertinti, ar puslapis vykdo klientinį Javascript atvaizdavimą & Nustatyti, ar HTML dokumentas naudoja Javascript bibliotekas, kurios naudojamos klientinio atvaizdavimo vieno puslapio žiniatinklio programose, jei taip -- nusiųsti tokį URL adresą į Javascript atvaizdavimo eilę  \\
    \hline
    U2.2 & Įvertinti etiškos puslapio peržiūros laiko intervalą & Atsižvelgti į HTTP užklausos serverio atsakymo laiką, domeno populiarumą, robots.txt nurodytą leistiną peržiūros intervalą \\
    \hline
    U2.3 & Atvaizduoti Javascript svetaines & Naudojant Javascript variklį atvaizduoti pilną HTML turinį svetainėms, kurios naudoja klientinio atvaizdavimo bibliotekas \\
    \hline
    \end{tabular}%
  \label{tab:use_case_table}%
\end{table}%