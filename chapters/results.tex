\sectionnonum{Rezultatai}

Darbe pasiekti \textbf{rezultatai}: 

\begin{enumerate}
    \item Atlikta saityno peržiūros robotų probleminės srities literatūrinė apžvalga: 
    \begin{itemize}
        \item identifikuoti saityno nuorodų ryšiai remiantis grafų teorijos sąvokomis
        \item pristatytas bazinis peržiūros robotų veikimo algoritmas
        \item atliktas saityno peržiūros robotų ir saityno duomenų surinkimo robotų sistemų palyginimas: išskirti panašumai ir trūkumai
        \item aprašytos 3 saityno peržiūros robotų veikimo politikos: \textbf{pasirinkimo}, \textbf{etiško žvalgymo} ir \textbf{pakartotinio apsilankymo}
        \item Identifikuotos 4 peržiūros robotų kategorijos: \textbf{plačiosios peržiūros}, \textbf{teminės peržiūros}, \textbf{inkrementinės peržiūros}, \textbf{išskirstytos peržiūros}
    \end{itemize}
    
    \item Apibrėžti saityno peržiūros robotų komponentai ir jų paskirtis: \textbf{peržiūros pasienis}, \textbf{HTTP parsiuntimo modulis}, \textbf{saityno nuorodų ištraukiklis}, \textbf{adresų skirstiklis} , \textbf{adresų filtras}, \textbf{dublikatų šalintojas}, \textbf{adresų prioritizuotojas}. Remiantis identifikuotais komponentais, UML veiklos diagrama apibrėžtas saityno peržiūros vykdymo ciklas
    
    
    \item Palygintos 3 saityno peržiūros robotų architektūros analizuojant peržiūros išplečiamumo ir darbo koordinavimo aspektus: \textbf{„PolyBot}, \textbf{„Mercator“}, \textbf{„Heritrix“} 
    
    
    \item Palygintas \textbf{tradicinis} ir \textbf{AJAX žiniatinklio atvaizdavimo} modeliai, identifikuotas saityno peržiūros robotų ribotumas žvalgyti AJAX tipo svetaines, pristatytas „Googlebot“ peržiūros roboto taikomas modelis atvaizduojant dinaminį turinį generuojančias svetaines
    
    
    \item Apibrėžta 12 funkcinių ir 3 nefunkciniai saityno peržiūros roboto prototipo reikalavimai
    
    
    \item Pristatytos prototipo įgyvendinumui pasirinktos „Azure“ debesų kompiuterijos tiekėjo technologijos: \textbf{„Service Fabric“ orchestravimo platofrma}, \textbf{„Service Bus“ eilių infrastruktūra}, \textbf{„Azure Table Storage NoSQL saugykla“}, \textbf{„Azure SQL“ duomenų bazė}. Argumentuoti pasirinkimo motyvai, identifikuotos nepasirinktos alternatyvos
    
    \item Įgyvendintas saityno peržiūros roboto prototipas:
    \begin{itemize}
        \item Apibrėžti 7 panaudojamumo atvejai, kurie atsekamumo matrica susieti su prototipui iškeltais reikalavimais
        \item Pristatyta realizuojant prototipą naudojama architektūra
        \item Nubrėžtos prototipo dinamiką vaizduojančios UML sekų ir veiklos diagramos, taip pat struktūrinė schema
        \item C\# kalba, naudojant .NET Core karkasą realizuotas apibrėžtos architektūros prototipas, aprašytos panaudotos bibliotekos
    \end{itemize}
    
    
    \item Atliktas prototipo vertinimas -- įvykdytas roboto veikimo stebėsenos eksperimentas, surinktos ir išanalizuotos funkcinės metrikos, taip pat prototipo komponentų apkrovos
\end{enumerate}