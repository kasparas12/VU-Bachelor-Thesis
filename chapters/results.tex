\sectionnonum{Rezultatai}

Darbe pasiekti \textbf{rezultatai}: 

\begin{enumerate}
    \item Atlikta saityno peržiūros robotų probleminės srities literatūrinė apžvalga: 
    \begin{itemize}
        \item  saityno nuorodų ryšiai identifikuoti remiantis grafų terminologijos analogijomis
        \item pristatytas bazinis peržiūros robotų veikimo algoritmas, leidžiantis robotui keliauti saityno nuorodų grafu
        \item atliktas saityno peržiūros robotų ir saityno duomenų surinkimo robotų sistemų palyginimas išskiriant panašumus ir trūkumus
        \item Aprašytos 3 pagrindinės literatūroje sutinkamos saityno peržiūros robotų veikimo politikos: \textbf{pasirinkimo}, \textbf{etiško žvalgymo} ir \textbf{pakartotinio apsilankymo}
        \item Identifikuotos 4 pagrindinės peržiūros robotų kategorijos: \textbf{plačiosios peržiūros}, \textbf{teminės peržiūros}, \textbf{inkrementinės peržiūros}, \textbf{išskirstytos peržiūros}
    \end{itemize}
    \item Apibrėžti pagrindiniai literatūroje sutinkami saityno peržiūros robotų komponentai: \textbf{peržiūros pasienis} \textbf{HTTP parsiuntimo modulis}, \textbf{saityno nuorodų ištraukiklis}, \textbf{adresų skirstiklis} , \textbf{adresų filtras}, \textbf{dublikatų šalintojas}, \textbf{adresų prioritizuotojas}. Remiantis identifikuotais komponentais, apibrėžtas saityno peržiūros ciklas
\end{enumerate}