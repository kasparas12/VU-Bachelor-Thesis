\sectionnonum{Įvadas}
    Saityno žvalgymo sistemos (angl. -- \textit{Web Crawling Systems}) atsirado kartu su pirmaisiais interneto paieškos varikliais ir išlieka esminis jų veikimo pagrindas \cite{WCArchitectureMicrosoft}. Tokios programos leidžia turinio agregavimo platformoms efektyviai identifikuoti naujus ar kintačius saityno resursus ir indeksuoti jų turinį, kurį vėliau gali pasiekti paieškos sistemų naudotojai, formuodami paieškos terminais grįstas užklausas. Saityno žvalgymo robotų pritaikymas neapsiriboja tik paieškos sistemomis, jie taip pat dažnai naudojami archyvuojant žiniatinklio resursus, identifikuojant autorių teises pažeidžiančias svetaines ar kenksmingus puslapius, taip pat atliekant didžiųjų duomenų gavybą \cite{WCArchitectureMicrosoft}.
    
    
    Žiniatinklio žvalgymo problema nėra nauja ir literatūroje analizuojama jau kelis dešimtmečius \cite{EffectiveWebCrawling}, \cite{MercatorLiterature}, nuo pat pirmųjų interneto naršyklių atsiradimo. Apie jos aktualumą verčia kalbėti eksponentinio Interneto resursų augimo tempai: 2020 duomenimis žiniatinklyje yra virš 1,7 mlrd svetainių. (iš kurių tik 12\% veikiančių ir pasiekiamų) \cite{InternetLiveStats}. Šis skaičius lyginant su 2010 statistika padidėjo net 8 kartus. Toks spartus žiniatinklio resursų kiekio augimas reikalauja lengvai išplečiamos sistemos architektūros, taip pat daug skaičiavimo ir talpinimo resursų: procesoriaus branduolių skaičiavimo galios, operatyviosios atminties, spartaus disko vietos ir didelio tinklo pralaidumo.
    
    
    Literatūroje absoliuti dauguma aprašytų viešų saityno peržiūros sistemų remiasi sudėtingos infrastruktūros nuoma ar pirkimu, jos paruošimu (aplinkų konfigūravimas) ir saityno peržiūros roboto sistemos paketo diegimu naudojantis komandinės eilutės instrukcijomis. Tai dažnai lemia didelius finansinius kaštus išlaikant resursus, taip pat specifinių infrastruktūros valdymo žinių poreikį. Mokslinėje literatūroje plačiai aprašyti vieši tokių sistemų sprendimai remiasi dešimtmečių senumo IT rinkos skaičiavimo galios, resursų talpinimo pajėgumų kontekstu, kuriame debesų kompiuterijos galimybės minimos minimaliai. Daugumoje tokių literatūroje aprašytų sprendimų remiasi centralizuotais sistemos komponentais, kurie plečiant sistemą lemia „butelio kaklelio“ efektą ir neleidžia pasiekti tiesinio pajėgumų išaugimo efekto. Komercinių saityno peržiūros sistemų (tokių kaip \textit{„Googlebot“}) architektūros yra slepiamos, nes lemia strateginę šių kompanijų sėkmę. Literatūroje taip pat stinga informacijos apie modernių saityno programų žvalgymo procesą, nors jis tik pasunkina padėtį, nes nuorodos tokiose svetainėse generuojamos dinamiškai, kai puslapis užkraunamas naršyklės, o tradiciniai peržiūros robotai nenaudoja Javascript atvaizdavimo variklio, todėl negali aptikti tokių resursų. Aprašytos problemos dėl apribotų resursų dažnai neleidžia mažesnėms įmonėms, startuoliams atlikti platesnio Interneto svetainių žvalgymo proceso ir užkerta kelią neišnaudotoms rinkos galimybėms. 
    
    Šiame darbe nagrinėjama galimybė saityno peržiūros sistemą realizuoti pasinaudojant viešo debesų kompiuterijos tiekėjo siūlomomis paslaugomis ir resursais. Tai leistų dinamiškai koreguoti tokių sistemų veikimo pajėgumus, mokėti už skaičiavimo laiką tik atliekant žvalgymo procesą (pay-as-you-go modelis). Būtų išvengiama sudėtingų infrastruktūros, duomenų centrų priežiūros procesų. Rašto darbe plačiai remiamasi Kalifornijos universiteto Mersedo miesto mokslininkų Mehdi Bahrami, Mukesh Singhal, Zixuan Zhuang atlika studija \cite{MercedCloudBasedWebCrawler} apie išskirstytą debesų kompiuterijos sprendimu grįstą saityno peržiūros sistemą ir naudojamasi pasiūlyta aukšto lygio tokio tipo sistemos architektūra. Pabrėžtina, jog nurodomoje studijoje nėra pasiekiama prototipinė realizacija, todėl darbe įgyvendinamas panašios, pakoreguotos architektūros saityno peržiūros roboto sistemos prototipas ir plačiau eksperimentiškai įvertinamos jo veikimo galimybės.
\\

\textbf{Darbo tikslas} -- įvertinti debesų kompiuterijos sprendimais pagrįsto modernių žiniatinklio programų peržiūros roboto panaudojamumo galimybes. 
\\

\textbf{Darbe keliami uždaviniai}:

\begin{enumerate}
    \item Apžvelgti saityno peržiūros robotų probleminę sritį
    \item Identifikuoti modernių žiniatinklio puslapių žvalgymo problemą ir rinkoje taikomus sprendimus
    \item Apibrėžti pagrindinius saityno peržiūros robotų struktūrinius komponentus
    \item Palyginti egzistuojančias peržiūros robotų architektūras
    \item Suformuluoti funkcinius ir nefunkcinius reikalavimus debesų kompiuterijos technologijomis pagrįstam prototipui
    \item Aprašyti naudojamas debesų kompiuterijos tiekėjo technologijas
    \item Sukurti ir aprašyti detalius peržiūros roboto prototipo architektūrinius sprendimus
    \item Įgyvendinti suprojektuotą prototipą
    \item Įvertinti sukurto prototipo saityno peržiūros efektyvumą, roboto komponentų apkrovas ir uždelsimus 
\end{enumerate}